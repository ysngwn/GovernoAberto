\chapter{Conclusão}

Através da análise isolada da transparência de dados do PTRF, disponibilizados pela Secretaria da Educação de São Paulo, é possível concluir que os responsáveis por desenhar a implementação de acessibilidade de dados através dos portais de transparência tiveram o cuidado, ou pelo menos uma preocupação, na forma como estes dados são disponibilizados, visto que quase um terço das boas práticas adotadas pelo W3C são seguidas. No entanto, ainda há muitos pontos a serem melhorados, principalmente no quesito de disponibilização dos dados de outras formas, por exemplo através de uma API ou em outros formatos abertos.

Além disto, ao realizar a análise comparativa com outros municípios brasileiros, constata-se que a transparência adotada por São Paulo não atende tantos requisitos quanto as capitais avaliadas. Por exemplo, o portal de transparência de Salvador, possui uma aba onde é possível baixar os metadados descritivos e estruturais, em outras palavras, há disponível uma documentação que explica quais são os dados disponibilizados. Do ponto de vista prático, é algo simples de ser feito e auxilia usuários leigos que buscam informação dos recursos destinados as escolas, mas que não foi ainda implementado no município de São Paulo. Por isso, é reitarado que há pontos que devem ser melhorados para que a transparência dos dados governamentais sejam mais acessíveis e entendíveis pelos paulistanos.

