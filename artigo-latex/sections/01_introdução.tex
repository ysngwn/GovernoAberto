\chapter{Introdução}

Diante do cenário de uma sociedade cada vez mais imersa na era digital e interativa, a gestão das escolas públicas não se esquiva dessa transformação. No meio de uma avalanche de informações e de expectativas em constante crescimento, a missão de prestar contas e democratizar os dados educacionais surge como uma jornada desafiadora, mas de vital importância, a ser percorrida. Apesar da relevância incontestável das escolas para o desenvolvimento integral dos alunos e para a consolidação das fundações socioeconômicas, estabelecer um sistema eficiente de prestação de contas continua sendo um desafio que temos pela frente.

Neste cenário, a democratização de dados educacionais surge como um protagonista crucial. A sociedade digital contemporânea, repleta de informações instantâneas e onipresentes, aguarda um acesso amplo e eficiente aos dados escolares. Esse passo, além de promover a transparência, é um impulso fundamental na direção da responsabilidade das escolas públicas. No entanto, o processo de democratização desses dados é como um caminho de duas vias, e ainda estamos navegando pelas curvas de sua compreensibilidade e acessibilidade.

O alicerce deste trabalho repousa sobre a teoria do Governo Aberto, que destaca a transparência, participação e colaboração como trampolim para uma governança eficaz. Essa teoria preconiza que, ao iluminar os cantos escuros dos dados governamentais, estamos incentivando a participação cidadã nas decisões públicas, proporcionando uma administração mais acertada. Levando esse preceito para o universo da educação, podemos projetar estratégias mais acuradas para aprimorar a prestação de contas e a democratização dos dados escolares.

