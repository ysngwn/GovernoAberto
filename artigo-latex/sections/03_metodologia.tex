\chapter{Metodologia}

Neste projeto, empregaremos uma metodologia que combina análise de conteúdo para avaliar a adesão das cidades selecionadas às Melhores Práticas para Dados na Web da W3C \cite{W3C}.

\section{Análise de Conteúdo}

Na análise de conteúdo, nos concentraremos nos portais de transparência de cada cidade para coletar e analisar os conjuntos de dados relacionados à educação. Esta etapa inclui o seguinte:

\begin{itemize}
    \item \textbf{Coleta de dados}: Visitaremos cada portal \cite{SP} \cite{CU} \cite{SA} \cite{MA} \cite{CG} e coletaremos os conjuntos de dados relevantes à educação disponíveis. Isso inclui dados sobre escolas, professores, alunos, desempenho dos alunos, financiamento e outros aspectos relevantes da educação.
    \item \textbf{Avaliação da qualidade dos dados}: Avaliaremos a qualidade dos dados coletados de acordo com os critérios estabelecidos pelas Melhores Práticas para Dados na Web da W3C \cite{W3CSUMMARY}. Isso inclui a análise da completude, precisão, relevância e consistência dos dados.
    \item \textbf{Avaliação da acessibilidade dos dados}: Avaliaremos a facilidade de acesso aos dados nos portais, a clareza das informações fornecidas e a presença de metadados que facilitam a compreensão e o uso dos dados.
\end{itemize}

No quesito da qualidade dos dados, é necessário definir qual será a medida específica adotada para cada ponto do W3C:

\begin{enumerate}

    \item \textbf{Fornecer metadados}: Verificado se o arquivo disponibilizado contém cabeçalho descritivo para cada coluna da planilha;
    
    \item \textbf{Fornecer metadados descritivos}: Verificado se é disponibilizado no arquivo ou em algum link da página visitada se há uma descrição que define especificamente o que as informações de cada coluna da planilha representa;

    \item \textbf{Fornecer metadados estruturais}: Verificado se é disponibilizado no arquivo ou em algum link da página visitada a estrutura que compõem os metadados da planilha;

    \item \textbf{Fornecer informações sobre a licença de dados}: Verificado se é disponibilizado um arquivo ou um link de acesso que descre a licença de dados;

    \item \textbf{Fornecer informações sobre a procedência dos dados}: Verificado se é informado como foi obtido os dados informados nos arquivos avaliados;

    \item \textbf{Fornecer informações de qualidade de dados}: Verificado se é informado se há dados faltantes, quantos se houver, o formato dos dados, etc;

    \item \textbf{Fornecer indicador de versão}: Exibição na página de download dos arquivos, ou no próprio arquivo, a versão dos dados disponibilizados;

    \item \textbf{Fornecer o histórico de versão}: Exibição de uma tabela contendo o histórico de versão. se houver indicação de versão;

    \item \textbf{Usar URIs persistentes como identificadores de conjuntos de dados}: Nos dados agrupados, fornecimento de URIs que levam aos dados específicos;

    \item \textbf{Usar URIs persistentes como identificadores dentro de conjunto de dados}: Nos dados específicos, exibição de URIs que levam a outros dados vinculados a este;

    \item \textbf{Atribuir URIs para as versões dos conjuntos de dados e séries}: Nos indicadores de versões, atribuir URIs que levam à planilha;

    \item \textbf{Usar formatos de dados padronizados legíveis por máquinas}: Os dados disponibilizados devem estar em arquivos csv, xlsx, xml ou outro formato que uma máquina consiga manipular os dados contidos;

    \item \textbf{Usar representações de dados que sejam independentes de localidade}: Dados como data, moedas e palavras são entendíveis independente da localidade do usuário;

    \item \textbf{Fornecer dados em formatos múltiplos}: Deve disponibilizar os dados em, ao menos, dois formatos diferentes;

    \item \textbf{Reutilizar vocabulários, dando preferência aos padronizados}: Os dados que são disponibilizados, se forem repetidos, devem estar de maneira igual, não com palavras semelhantes ou formatos diferentes;

    \item \textbf{Escolher o nível de formalização adequado}: Ser retirado ou colocado formato de formalização dos dados que melhor define o que está sendo representado pelas informações exibidas;

    \item \textbf{Fornecer download em massa}: Permitir, através de diferentes de usuários, ou do mesmo, a extração dos dados sem bloqueio por quantidade de dados;

    \item \textbf{Fornecer subjconjuntos para conjuntos de dados extensos}: Poder desagrupar dados que são construídos a partir de outros dados;

    \item \textbf{Usar negociação de conteúdo para disponibilizar dados em formatos múltiplos}: Em uma API, receber os tipos de formatos aceitos pelo cliente e disponibilizar os dados em um dos formatos recebidos;

    \item \textbf{Fornecer acesso em tempo real}: Ao tentar realizar o download dos dados, deve ser executado o procedimento no mesmo momento, não deve ser enviado por e-mail ou por notificação após x minutos;

    \item \textbf{Fornecer dados atualizados}: Os dados são atualizados diariamente, de segunda a sexta, desconsiderando feriados;

    \item \textbf{Fornecer uma explicação para os dados que não estão disponíveis}: Se houver dados faltantes, conter uma explicação dentro de uma aba de observações ou na página de download;

    \item \textbf{Disponibilizar dados por meio de uma API}: Os dados podem ser obtidos atraves de uma requisição em um endpoint de uma API;

    \item \textbf{Usar padrões Web como base para construção de APIs}: Utilizar um estilo arquitetônico baseado nas tecnologias da própria Web;

    \item \textbf{Fornecer documentação completa para as APIs}: Fornecer informações completas na Web sobre a API. Atualizar a documentação conforme características adicionadas ou modificações realizadas;

    \item  \textbf{Evitar alterações que afetem o funcionamento de sua API}: Ao realizar uma requisição para a API recentemente atualizada, os dados devem serem idênticos aos obtidos antes da atualização

    \item \textbf{Preservar identificadores}: Os identificadores dos dados disṕonibilizados não devem ser faltantes no arquivo

    \item \textbf{Avaliar a cobertura do conjunto de dados}: Os dados devem estar disponíveis independente da localidade do usuário;

    \item \textbf{Coletar feedback de consumidores de dados}: Deve haver uma aba para avaliação ou comentário na página de download do arquivo;

    \item \textbf{Compartilhar o feedback disponível}: Deve haver uma aba que disponibize as avaliações de usuários terceiros;

    \item \textbf{Enriquecer dados por meio da geração de novos dados}: Dados secundários devem ser obtidos por meio de dados primários, ambos disponbilizados no arquivo;

    \item \textbf{Fornecer visualizações complementares}: Fornecer visualizações complementares, como gráficos, grids, mapas, heat maps, entre outras formas de exibição;

    \item \textbf{Fornecer feedback para o publicador original}: Fornecer feedback para o responsável dos dados disponibilizados;

    \item \textbf{Obedecer os termos de licença}: Se houver disponibilizado os termos de licença, não infrigí-los;

    \item \textbf{Citar a publicação original do conjunto de dados}: Se for um dado republicado, citar a publicação original por meio de referências.

\end{enumerate}

\section{Comparação e Recomendações}

Após a análise de conteúdo, compararemos nossas descobertas entre as diferentes cidades. Isto nos permitirá identificar as cidades que estão liderando em termos de práticas de dados abertos e aquelas que podem precisar de melhorias.

Com base em nossas descobertas, formularíamos recomendações para cada cidade, sugerindo maneiras de melhorar a transparência, a qualidade e a acessibilidade dos dados de educação.

Os resultados desta análise permitirão não apenas uma avaliação aprofundada das práticas de dados abertos nos municípios selecionados, mas também fornecerão uma base para recomendações que possam promover uma maior transparência e acessibilidade dos dados de educação no futuro.

