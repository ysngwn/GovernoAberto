\section{Programa de Transferência de Recursos Financeiros - PTRF}

O Programa de Transferência de Recursos Financeiros (PTRF) \cite{SP} é uma iniciativa da Prefeitura de São Paulo para dar maior autonomia financeira às escolas públicas da cidade. O programa fornece recursos para que as próprias escolas possam gerenciar e investir em melhorias e necessidades específicas.

O site de dados abertos da Prefeitura de São Paulo é um repositório que contém informações e conjuntos de dados sobre o PTRF. Isso pode incluir detalhes sobre quanto dinheiro foi alocado para diferentes escolas, como esses fundos foram gastos e talvez informações sobre os resultados do programa.

\subsection{Avaliação de Usabilidade}

A Avaliação Heurística envolveu um pequeno conjunto de avaliadores (integrantes do grupo) que examinaram a interface do sistema e comparam seus elementos à lista de heurísticas.

\textbf{Visibilidade do status do sistema}: O site atualiza os usuários sobre o progresso de ações em tempo real, por exemplo, ao clicar um arquivo, uma interface de carregamento é exibida.

\textbf{Correspondência entre o sistema e o mundo real}: O portal não utiliza terminologia técnica, tornando-o de fácil compreensão para usuários não técnicos.

\textbf{Controle do usuário e liberdade}: O site não oferece uma opção "voltar" ou "cancelar" claramente visível em todas as páginas, permitindo que os usuários saiam de situações indesejadas sem ter que usar a função de volta do navegador.

\textbf{Consistência e padrões}: O portal é consistentemente projetado com o mesmo esquema de cores e formatos de tabela, criando um ambiente familiar para os usuários.

\textbf{Prevenção de erros}: O site possui apenas um campo de pesquisa como entrada do usuário, mas não apresenta qualquer tipo de prevenção de erros.

\textbf{Reconhecimento em vez de lembrança}: As ações e opções disponíveis para o usuário são visíveis e claramente marcadas. O usuário não precisa lembrar informações de uma parte do site para outra.

\textbf{Flexibilidade e eficiência de uso}: O site tem recursos que permitem o uso eficiente tanto por usuários novatos quanto por usuários experientes.

\textbf{Design minimalista}: As páginas do site não estão sobrecarregadas de informações. Cada informação adicional é relevante e suporta as tarefas do usuário.

\textbf{Ajuda os usuários a reconhecer, diagnosticar e corrigir erros}: Se uma pesquisa não retornar resultados, o sistema responde com ``Por favor, tente um novo termo de pesquisa.''

\textbf{Ajuda e documentação}: O site não tem um FAQ ou um guia do usuário para ajudar os usuários a entender como usar o site e suas funções.

\subsection{Análise Documental}

